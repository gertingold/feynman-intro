\documentclass[11pt,DIV=19,parskip=half]{scrartcl}
\usepackage{amsmath}
\usepackage{fontawesome}
\usepackage{url}
\usepackage{hyperref}
\hypersetup{%
  pdftitle={Introduction to Feynman path integrals}
  ,pdfauthor={Gert-Ludwig Ingold}
  ,pdfsubject={Exercise for the Bad Honnef Physics School on
               Methods of Path Integration in Modern Physics, Bad Honnef, 25.8.2019}
  ,pdfkeywords={Feynman path integrals, harmonic oscillator, Mehler formula}
}
\begin{document}
\pagestyle{empty}
\begin{footnotesize}
Bad Honnef Physics School \textit{Methods of Path Integration in Modern Physics},
August 25--31, 2019\\
Gert-Ludwig Ingold: Introduction to Feynman path integrals\\
\faicon{github} \url{https://github.com/gertingold/feynman-intro}\\
\vbox{\hrulefill}
\end{footnotesize}


\vspace{0.5truecm}
\textbf{Exercise:} Check the choice of phases in the propagator of the harmonic
oscillator
\begin{equation}
 K(q_\text{f}, q_\text{i}, t)
      = \sqrt{\frac{m\omega}{2\pi\hbar\vert{\sin(\omega t)}\vert}}
         \exp\left(\text{i}\frac{m\omega}{2\hbar}\frac{(q_\text{f}^2+q_\text{i}^2)\cos(\omega t)
         -2q_\text{i}q_\text{f}}{\sin(\omega t)}-\text{i}\left(\frac{\pi}{4}+n\frac{\pi}{2}\right)\right)
\end{equation}
with $n=\lfloor\omega t/\pi\rfloor$ by combining two propagators over time intervals
$3\pi/4\omega$ to a propagator over the time interval $3\pi/2\omega$ by means of the semigroup
property.

\vspace{0.5truecm}
\textbf{Solution:}

Inserting the explicit values for the time intervals, we have
\begin{align}
 K(q_\text{f}, q_\text{i}, \frac{3\pi}{4\omega})
      &= \sqrt{\frac{m\omega}{\sqrt{2}\pi\hbar}}
         \exp\!\left[\text{i}\frac{m\omega}{2\hbar}\left(-(q_\text{f}^2+q_\text{i}^2)
         -2\sqrt{2}q_\text{i}q_\text{f}\right)-\text{i}\frac{\pi}{4}\right]\\
\intertext{and}
 K(q_\text{f}, q_\text{i}, \frac{3\pi}{2\omega})
      &= \sqrt{\frac{m\omega}{2\pi\hbar}}
         \exp\!\left[\text{i}\frac{m\omega}{\hbar}q_\text{i}q_\text{f}-\text{i}\frac{3\pi}{4}\right]
\end{align}
By means of the semigroup property, we get
\begin{equation}
 \begin{aligned}
  K(q_\text{f}, q_\text{i}, \frac{3\pi}{2\omega}) &= \int_{-\infty}^{+\infty}\text{d}q'
                K(q_\text{f}, q', \frac{3\pi}{4\omega}) K(q', q_\text{i}, \frac{3\pi}{4\omega})\\
   &= \frac{m\omega}{\sqrt{2}\pi\hbar} \int_{-\infty}^{+\infty}\text{d}q'
         \exp\!\left[\text{i}\frac{m\omega}{2\hbar}\left(-q_\text{f}^2-q_\text{i}^2-2q'^2
                        -2\sqrt{2}(q_\text{f}+q_\text{i})q'\right)-\text{i}\frac{\pi}{2}\right]\\
   &= \frac{m\omega}{\sqrt{2}\pi\hbar} 
         \exp\!\left[-\text{i}\frac{m\omega}{2\hbar}(q_\text{f}^2+q_\text{i}^2)-\text{i}\frac{\pi}{2}\right]
         \int_{-\infty}^{+\infty}\text{d}q'
         \exp\!\left[-\text{i}\frac{m\omega}{\hbar}
                 \left(q'+\frac{1}{\sqrt{2}}(q_\text{f}+q_\text{i})\right)^2\right]\\
   &\hspace{9truecm}\times\exp\!\left[\text{i}\frac{m\omega}{2\hbar}(q_\text{f}+q_\text{i})^2\right]\,.
 \end{aligned}
\end{equation}
The Fresnel integral becomes
\begin{equation}
 \int_{-\infty}^{+\infty}\text{d}q'\exp\!\left[-\text{i}\frac{m\omega}{\hbar}
      \left(q'+\frac{1}{\sqrt{2}}(q_\text{f}+q_\text{i})\right)^2\right]
 = \sqrt{\frac{\pi\hbar}{m\omega}}\exp\!\left(-\text{i}\frac{\pi}{4}\right)\,.
\end{equation}
Thus, we finally arrive at the expected result
\begin{equation}
 K(q_\text{f}, q_\text{i}, \frac{3\pi}{2\omega}) = \sqrt{\frac{m\omega}{2\pi\hbar}}
         \exp\!\left[\text{i}\frac{m\omega}{\hbar}q_\text{i}q_\text{f}-\text{i}\frac{3\pi}{4}\right]\,.
\end{equation}

\end{document}
