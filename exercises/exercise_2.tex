\documentclass[11pt,DIV=19,parskip=half]{scrartcl}
\usepackage{amsmath}
\usepackage{fontawesome}
\usepackage{url}
\usepackage{hyperref}
\hypersetup{%
  pdftitle={Introduction to Feynman path integrals}
  ,pdfauthor={Gert-Ludwig Ingold}
  ,pdfsubject={Exercise for the Bad Honnef Physics School on
               Methods of Path Integration in Modern Physics, Bad Honnef, 25.8.2019}
  ,pdfkeywords={Feynman path integrals, propagator of free particle, solution by discretization}
}
\begin{document}
% \pagestyle{empty}
\begin{footnotesize}
Bad Honnef Physics School \textit{Methods of Path Integration in Modern Physics},
August 25--31, 2019\\
Gert-Ludwig Ingold: Introduction to Feynman path integrals\\
\faicon{github} \url{https://github.com/gertingold/feynman-intro}\\
\vbox{\hrulefill}
\end{footnotesize}


\vspace{0.5truecm}
\textbf{Exercise:} Evaluate the Feynman path integral for the free particle by means of
discretization. Why is the result independent of the number $N$ of discretization steps?


\vspace{0.5truecm}
\textbf{Solution:}

The discretized version of the propagator of a free particle is given by
\begin{equation}
 \label{eq:discretizedpropagator}
 K(q_\text{f}, q_\text{i}, t)
 = \sqrt{\frac{m}{2\pi\text{i}\hbar\Delta t}}\int_{-\infty}^{+\infty}
   \prod_{n=1}^{N-1}\left(\sqrt{\frac{m}{2\pi\text{i}\hbar\Delta t}}\text{d}q_n\right)
   \exp\left(\frac{\text{i}}{\hbar}\sum_{n=1}^N\frac{m}{2}\frac{(q_n-q_{n-1})^2}{\Delta t}\right)\,.
\end{equation}
Here, $q_N=q_\text{f}$ and $q_0=q_\text{i}$.

For a multidimensional Gaussian integral, we have
\begin{equation}
 \label{eq:ndgaussian}
 \int_{-\infty}^{+\infty}\text{d}x^N \text{e}^{-\mathbf{x}^\text{T}\mathbf{A}\mathbf{x}} =
                \sqrt{\frac{\pi^N}{\det(\mathbf{A})}}\,.
\end{equation}
Without loss of generality, the matrix $\mathbf{A}$ can be assumed to be symmetric, because
an antisymmetric matrix will not contribute to the bilinear form. The relation (\ref{eq:ndgaussian})
can then be proven by transforming into the eigenbasis of $\mathbf{A}$ and carrying out the individual
integrals. The right-hand side is then obtained by realizing that the product of eigenvalues equals
the determinant of the corresponding matrix.

The same line of reasoning can be applied to a Fresnel integral and we find
\begin{equation}
 \label{eq:ndfresnel}
 \int_{-\infty}^{+\infty}\text{d}x^N \text{e}^{\text{i}\mathbf{x}^\text{T}\mathbf{A}\mathbf{x}} =
                \sqrt{\frac{(\text{i}\pi)^N}{\det(\mathbf{A})}}\,.
\end{equation}
We thus aim at bringing (\ref{eq:discretizedpropagator}) into such a form.

In a first step, we consider the sum in the exponent.
\begin{equation}
 \sum_{n=1}^N(q_n-q_{n-1})^2 = q_{N-1}^2+q_1^2+\sum_{n=2}^{N-1}(q_n-q_{n-1})^2
                               -2q_\text{f}q_{N-1}-2q_\text{i}q_1+q_\text{f}^2+q_\text{i}^2\,.
\end{equation}
Realizing that all squares of $q_i, i=1,\ldots,N-1$ appear twice and that the mixed terms arising
from the squares in the sum have to be splitted into two contributions in a bilinear form, we
obtain
\begin{equation}
 \label{eq:sum_of_squares}
 \sum_{n=1}^N(q_n-q_{n-1})^2 = \mathbf{x}^T\mathbf{A}\mathbf{x}-2\mathbf{b}^T\mathbf{x}
                               +q_\text{f}^2+q_\text{i}^2\,.
\end{equation}
Here, we have introduced the tridiagonal $(N-1)\times(N-1)$ matrix
\begin{equation}
 \label{eq:matrixA}
 \mathbf{A} = \begin{pmatrix}
               2     & -1     &  0     & \cdots & 0\\
              -1     &  2     & \ddots & \ddots & \vdots\\
               0     & \ddots & \ddots & \ddots & 0\\
              \vdots & \ddots & \ddots &  2     & -1\\
               0     & \cdots &  0     & -1     & 2

              \end{pmatrix}
\end{equation}
and the vectors
\begin{equation}
 \mathbf{x} = \begin{pmatrix} q_{N-1} \\ q_{N-2} \\ \vdots \\ q_2 \\ q_1\end{pmatrix}\qquad
 \mathbf{b} = \begin{pmatrix} q_\text{f} \\ 0 \\ \vdots \\ 0 \\ q_\text{i}\end{pmatrix}
\end{equation}
In (\ref{eq:sum_of_squares}) we can now complete the square to obtain
\begin{equation}
 \sum_{n=1}^N(q_n-q_{n-1})^2 = (\mathbf{x}^T-\mathbf{b}^T\mathbf{A}^{-1})\mathbf{A}
   (\mathbf{x}-\mathbf{A}^{-1}\mathbf{b})-\mathbf{b}^T\mathbf{A}^{-1}\mathbf{b}+q_\text{f}^2+q_\text{i}^2\,.
\end{equation}
Carrying out the Fresnel integrals with the help of (\ref{eq:ndfresnel}) we find as a first result
\begin{equation}
 \label{eq:discretizedpropagatorafterintegration}
 K(q_\text{f}, q_\text{i}, t) = \sqrt{\frac{m}{2\pi\text{i}\hbar\Delta t\det(\mathbf{A})}}
    \exp\left(\frac{\text{i}m}{2\hbar\Delta t}
              (q_\text{f}^2+q_\text{i}^2-\mathbf{b}^T\mathbf{A}^{-1}\mathbf{b})\right)\,.
\end{equation}

In order to proceed further, we need to evaluate the determinant of the matrix $\mathbf{A}$ defined in
(\ref{eq:matrixA}). One possibility is to realize that $\mathbf{A}$ is a tridiagonal Toeplitz matrix
for which the eigenvalues are known analytically. For our special case, the eigenvalues read
\begin{equation}
 \lambda_k = 2\left(1+\cos\left(\frac{\pi k}{N}\right)\right)\qquad k=1,\ldots, N-1
\end{equation}
The determinant $D_{N-1}$ of the $(N-1)\times(n-1)$ matrix $\mathbf{A}$ then becomes
\begin{equation}
 \label{eq:determinantA}
 \begin{aligned}
  D_{N-1} = \det(\mathbf{A}) &= \prod_{k=1}^{N-1}2\left(1+\cos\left(\frac{\pi k}{N}\right)\right)\\
                             &= \lim_{x\to -1}\prod_{k=1}^{N-1}\left(x^2-2x\cos\left(\frac{\pi k}{N}\right)
                                                                     +1\right)\\
                             &= \lim_{x\to -1}\frac{x^{2N}-1}{x^2-1}\\
                             &= N\,.
 \end{aligned}
\end{equation}
This result can also be obtained by induction. The determinant of $\mathbf{A}$ satisfies
the recursion relation
\begin{equation}
 D_n = 2D_{n-1}-D_{n-2}
\end{equation}
with the initial conditions
\begin{equation}
 D_1 = 2, D_2 = 3\,.
\end{equation}
It can easily be checked that this recursion problem is indeed solved by $D_{N-1}=N$.

In view of the exponent of (\ref{eq:discretizedpropagatorafterintegration}), we need the
entries in the corners of the matrix $\mathbf{A}^{-1}$ which can easily be obtained by
directly determining the corresponding elements of the adjugate matrix. On the diagonal, we find
\begin{equation}
 (\mathbf{A}^{-1})_{11} = (\mathbf{A}^{-1})_{N-1,N-1} = \frac{D_{N-2}}{D_{N-1}} = 1-\frac{1}{N}\,.
\end{equation}
In the off-diagonal corners, the adjugate matrix takes the value 1 and we obtain
\begin{equation}
 (\mathbf{A}^{-1})_{N-1,1} = (\mathbf{A}^{-1})_{1,N-1} = \frac{1}{D_{N-1}} = \frac{1}{N}\,.
\end{equation}
In the exponent of (\ref{eq:discretizedpropagatorafterintegration}), we thus have
\begin{equation}
 \mathbf{b}^T\mathbf{A}^{-1}\mathbf{b} = \left(1-\frac{1}{N}\right)(q_\text{f}^2+q_\text{i}^2)
                                         +\frac{2}{N}q_\text{f}q_\text{i}
   = q_\text{f}^2+q_\text{i}^2 - \frac{(q_\text{f}-q_\text{i})^2}{N}
\end{equation}

Inserting this result together with (\ref{eq:determinantA}) into
(\ref{eq:discretizedpropagatorafterintegration}), we arrive at
\begin{equation}
 K(q_\text{f}, q_\text{i}, t) = \sqrt{\frac{m}{2\pi\text{i}\hbar N\Delta t}}
    \exp\left(\frac{\text{i}m}{2\hbar}\frac{(q_\text{f}-q_\text{i})^2}{N\Delta t}\right)\,.
\end{equation}
Since we had divided the original time interval of length $t$ into $N$ equidistant pieces,
we have $\Delta t=t/N$ and thus find the exact propagator of the free particle
\begin{equation}
 K(q_\text{f}, q_\text{i}, t) = \sqrt{\frac{m}{2\pi\text{i}\hbar t}}
    \exp\left(\frac{\text{i}m}{2\hbar}\frac{(q_\text{f}-q_\text{i})^2}{t}\right)\,.
\end{equation}

The fact that we obtain the correct result for any value of $N$ and do not need to take
the limit $N\to\infty$ is exceptional. The reason is that for vanishing potential energy $V=0$,
the Lie-Trotter formula is exact not only in the limit $N\to\infty$ but also for any finite $N$.
Another way of viewing it is that the decomposition in the initial expression
(\ref{eq:discretizedpropagator}) is nothing else than a multiple application of the 
semigroup property which we had proven in another exercise. 

\end{document}
