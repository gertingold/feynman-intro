\documentclass[11pt,DIV=19,parskip=half]{scrartcl}
\usepackage{amsmath}
\usepackage{fontawesome}
\usepackage{url}
\usepackage{hyperref}
\hypersetup{%
  pdftitle={Introduction to Feynman path integrals}
  ,pdfauthor={Gert-Ludwig Ingold}
  ,pdfsubject={Exercise for the Bad Honnef Physics School on
               Methods of Path Integration in Modern Physics, Bad Honnef, 25.8.2019}
  ,pdfkeywords={Feynman path integrals, particle on a ring, change of representation}
}
\begin{document}
% \pagestyle{empty}
\begin{footnotesize}
Bad Honnef Physics School \textit{Methods of Path Integration in Modern Physics},
August 25--31, 2019\\
Gert-Ludwig Ingold: Introduction to Feynman path integrals\\
\faicon{github} \url{https://github.com/gertingold/feynman-intro}\\
\vbox{\hrulefill}
\end{footnotesize}


\vspace{0.5truecm}
\textbf{Exercise:} Show that the propagator for a particle on a ring
\begin{equation}
 \label{eq:propagator1}
 K(\phi_\text{f}, \phi_\text{i}, t) = R\sqrt{\frac{m}{2\pi\text{i}\hbar t}}
    \sum_{n=-\infty}^{+\infty}\exp\!\left(\frac{\text{i}}{\hbar}\frac{mR^2}{2}
			      \frac{(\phi_\text{f}-\phi_\text{i}-2\pi n)^2}{t}\right)
\end{equation}
obtained from the Feynman path integral is equivalent to the representation
\begin{equation}
 \label{eq:propagator2}
 K(\phi_\text{f}, \phi_\text{i}, t) = \frac{1}{2\pi}\sum_{\ell=-\infty}^{+\infty}
	\exp\!\left(\text{i}\ell(\phi_\text{f}-\phi_\text{i})
               -\text{i}\frac{\hbar\ell^2}{2mR^2}t\right)
\end{equation}
obtained by means of the eigenvalues and eigenfunctions of the Hamiltonian.

\vspace{0.5truecm}
\textbf{Solution:}

The central step in going from (\ref{eq:propagator1}) to (\ref{eq:propagator2}) involves
the Fourier representation of the $\delta$-comb
\begin{equation}
 \label{eq:delta_comb}
 \sum_{n=-\infty}^{+\infty}\delta(x-2\pi n)
     = \frac{1}{2\pi}\sum_{\ell=-\infty}^{+\infty}\text{e}^{\text{i}\ell x}\,.
\end{equation}

The propagator is a $2\pi$-periodic function in $q_\text{f}-q_\text{i}$ so that we can
express the sum in (\ref{eq:propagator1}) in terms of an integral over a $\delta$-comb:
\begin{equation}
 K(\phi_\text{f}, \phi_\text{i}, t) = R\sqrt{\frac{m}{2\pi\text{i}\hbar t}}
   \int_{-\infty}^{+\infty}\!\!\text{d}\phi\exp\!\left(\frac{\text{i}}{\hbar}\frac{mR^2}{2t}\phi^2\right)
   \sum_{n=-\infty}^{+\infty}\delta\big(\phi-(\phi_\text{f}-\phi_\text{i}-2\pi n)\big)\,.
\end{equation}
According to (\ref{eq:delta_comb}), we can write
\begin{equation}
 \sum_{n=-\infty}^{+\infty}\delta\big(\phi-(\phi_\text{f}-\phi_\text{i}-2\pi n)\big) =
   \frac{1}{2\pi}\sum_{\ell=-\infty}^{+\infty}\text{e}^{\text{i}\ell(\phi-\phi_\text{f}+\phi_\text{i})}
\end{equation}
and thus obtain
\begin{equation}
 K(\phi_\text{f}, \phi_\text{i}, t) = \frac{R}{2\pi}\sqrt{\frac{m}{2\pi\text{i}\hbar t}}
  \sum_{\ell=-\infty}^{+\infty}\text{e}^{-\text{i}(\phi_\text{f}-\phi_\text{i})}
  \int_{-\infty}^{+\infty}\!\!\text{d}\phi\exp\!\left(\frac{\text{i}}{\hbar}\frac{mR^2}{2t}\phi^2
	                  +\text{i}\ell\phi\right)\,.
\end{equation}
Completing the square in the exponent of the integrand, we can evaluate the Fresnel integral.
\begin{equation}
 \begin{aligned}
  K(\phi_\text{f}, \phi_\text{i}, t) &= \frac{R}{2\pi}\sqrt{\frac{m}{2\pi\text{i}\hbar t}}
  \sum_{\ell=-\infty}^{+\infty}\text{e}^{-\text{i}(\phi_\text{f}-\phi_\text{i})}
	 \int_{-\infty}^{+\infty}\!\!\text{d}\phi\exp\!\left[\text{i}\frac{mR^2}{2\hbar t}
			\left(\phi+\frac{\hbar t}{mR^2}\ell\right)^2\right]
			\exp\left(-\text{i}\frac{\hbar t}{2mR^2}\ell^2\right)\\
   &= \frac{1}{2\pi}\sum_{\ell=-\infty}^{+\infty}\exp\left(-\text{i}(\phi_\text{f}-\phi_\text{i})\ell
			-\text{i}\frac{\hbar t}{2mR^2}\ell^2\right)\\
 \end{aligned}
\end{equation}
Replacing the summation index $\ell$ by $-\ell$, we arrive at the desired result (\ref{eq:propagator2}).

\end{document}
