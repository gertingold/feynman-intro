\documentclass[11pt,DIV=19,parskip=half]{scrartcl}
\usepackage{amsmath}
\usepackage{fontawesome}
\usepackage{url}
\usepackage{hyperref}
\hypersetup{%
  pdftitle={Introduction to Feynman path integrals}
  ,pdfauthor={Gert-Ludwig Ingold}
  ,pdfsubject={Exercise for the Bad Honnef Physics School on
               Methods of Path Integration in Modern Physics, Bad Honnef, 25.8.2019}
  ,pdfkeywords={Feynman path integrals, particle in the box, change of representation}
}
\begin{document}
% \pagestyle{empty}
\begin{footnotesize}
Bad Honnef Physics School \textit{Methods of Path Integration in Modern Physics},
August 25--31, 2019\\
Gert-Ludwig Ingold: Introduction to Feynman path integrals\\
\faicon{github} \url{https://github.com/gertingold/feynman-intro}\\
\vbox{\hrulefill}
\end{footnotesize}


\vspace{0.5truecm}
\textbf{Exercise:} Show that the propagator for a particle in an infinitely deep potential well
\begin{equation}
 \label{eq:propagator1}
 K(q_\text{f}, q_\text{i}, t) = \sqrt{\frac{m}{2\pi\text{i}\hbar t}}\sum_{n=-\infty}^{+\infty}
  \left[\exp\!\left(\frac{\text{i}}{\hbar}\frac{m(2nL+q_\text{f}-q_\text{i})^2}{2t}\right)
  -\exp\!\left(\frac{\text{i}}{\hbar}\frac{m(2nL-q_\text{f}-q_\text{i})^2}{2t}\right)\right]
\end{equation}
obtained from the Feynman path integral agrees with the representation
\begin{equation}
 \label{eq:propagator2}
 K(q_\text{f}, q_\text{i}, t) = \frac{2}{L}\sum_{j=1}^{\infty}
        \sin\!\left(\frac{\pi j}{L}q_\text{f}\right)\sin\!\left(\frac{\pi j}{L}q_\text{i}\right)
   \exp\left(-\text{i}\frac{\hbar\pi^2j^2}{2mL^2}t\right)
\end{equation}
based on the eigenfunctions and eigenvalues of the Hamiltonian.

\vspace{0.5truecm}
\textbf{Solution:}

We follow the same strategy as for the propagator for a particle on a ring and consider the
first part of the sum in (\ref{eq:propagator1}). The second part then immediately follows by
the replacement $q_\text{f}\to-q_\text{f}$.

While for the particle on a ring, we were
dealing with a $2\pi$-periodic function, we now consider a function of period $2L$. The Fourier
representation of the corresponding $\delta$-comb is given by
\begin{equation}
 \sum_{n=-\infty}^{+\infty}\delta(x-2Ln) = \frac{1}{2L}\sum_{\ell=-\infty}^{+\infty}
                                              \exp\!\left(\text{i}\frac{\pi}{L}\ell x\right)\,.
\end{equation}

Going through the same steps as for a particle on a ring, we have
\begin{equation}
 \begin{aligned}
  \sum_{n=-\infty}^{+\infty}\exp\!\left(\frac{\text{i}}{\hbar}
                 \frac{m(2nL+q_\text{f}-q_\text{i})^2}{2t}\right) &= 
  \sum_{n=-\infty}^{+\infty}\int_{-\infty}^{+\infty}\text{d}q\,\delta(q-q_\text{f}+q_\text{i}-2nL)
        \exp\!\left(\frac{\text{i}}{\hbar}\frac{mq^2}{2t}\right)\\
  &= \frac{1}{2L}\sum_{\ell=-\infty}^{+\infty}\int_{-\infty}^{+\infty}\text{d}q
       \exp\!\left(\text{i}\frac{\pi}{L}\ell(q-q_\text{f}+q_\text{i})\right)
       \exp\!\left(\frac{\text{i}}{\hbar}\frac{mq^2}{2t}\right)\\
  &= \frac{1}{2L}\sum_{\ell=-\infty}^{+\infty}
         \exp\!\left(\text{i}\frac{\pi}{L}\ell(q_\text{f}-q_\text{i})\right)
  \int_{-\infty}^{+\infty}\text{d}q\exp\!\left(\frac{\text{i}}{\hbar}\frac{mq^2}{2t}
                         -\text{i}\frac{\pi}{L}\ell q\right)\,.
 \end{aligned}
\end{equation}
In the last line, we have replaced the summation index $\ell$ by $-\ell$. The Fresnel integral
can be evaluated to yield
\begin{equation}
 \begin{aligned}
  \int_{-\infty}^{+\infty}\text{d}q\exp\!\left(\frac{\text{i}}{\hbar}\frac{mq^2}{2t}
          -\text{i}\frac{\pi}{L}\ell q\right) &=
  \int_{-\infty}^{+\infty}\text{d}q\exp\!\left[\text{i}\frac{m}{2\hbar t}\left(q-
            \frac{\hbar\pi\ell}{mL}t\right)^2\right]
        \exp\!\left(-\text{i}\frac{\hbar\pi^2\ell^2}{2mL^2}t\right)\\
    &= \sqrt{\frac{2\pi\text{i}\hbar t}{m}}\exp\!\left(-\text{i}\frac{\hbar\pi^2\ell^2}{2mL^2}t\right)\,.
 \end{aligned}
\end{equation}
Inserting this result into the previous expression, we obtain
\begin{equation}
 \sqrt{\frac{m}{2\pi\text{i}\hbar t}} \sum_{n=-\infty}^{+\infty}\exp\!\left(\frac{\text{i}}{\hbar}
                 \frac{m(2nL+q_\text{f}-q_\text{i})^2}{2t}\right) = 
 \frac{1}{2L}\sum_{\ell=-\infty}^{+\infty}
    \exp\!\left(\text{i}\frac{\pi}{L}\ell(q_\text{f}-q_\text{i})
       -\text{i}\frac{\hbar\pi^2\ell^2}{2mL^2}t\right)\,.
\end{equation}
Replacing $q_\text{f}$ by $-q_\text{f}$, we find for the second contribution to (\ref{eq:propagator1})
\begin{equation}
 \sqrt{\frac{m}{2\pi\text{i}\hbar t}} \sum_{n=-\infty}^{+\infty}\exp\!\left(\frac{\text{i}}{\hbar}
                 \frac{m(2nL-q_\text{f}-q_\text{i})^2}{2t}\right) = 
 \frac{1}{2L}\sum_{\ell=-\infty}^{+\infty}
    \exp\!\left(\text{i}\frac{\pi}{L}\ell(-q_\text{f}-q_\text{i})
       -\text{i}\frac{\hbar\pi^2\ell^2}{2mL^2}t\right)\,.
\end{equation}
Subtracting the two contributions, we realize that the terms depending on $q_\text{f}$ can be combined
to a sine function
\begin{equation}
 K(q_\text{f}, q_\text{i}, t) = \frac{\text{i}}{L}\sum_{\ell=-\infty}^{+\infty}
     \sin\!\left(\frac{\pi}{L}\ell q_\text{f}\right)
    \exp\!\left(-\text{i}\frac{\pi}{L}\ell q_\text{i}-\text{i}\frac{\hbar\pi^2\ell^2}{2mL^2}t\right)\,.
\end{equation}
Since the term with $\ell=0$ vanishes, we can rewrite this expression as
\begin{equation}
 \begin{aligned}
  K(q_\text{f}, q_\text{i}, t) &= \frac{\text{i}}{L}\sum_{\ell=1}^{+\infty}
      \sin\!\left(\frac{\pi}{L}\ell q_\text{f}\right)\left[
             \exp\!\left(-\text{i}\frac{\pi}{L}\ell q_\text{i}\right)
             -\exp\!\left(\text{i}\frac{\pi}{L}\ell q_\text{i}\right)\right]
     \exp\!\left(-\text{i}\frac{\hbar\pi^2\ell^2}{2mL^2}t\right)\\
  &= \frac{2}{L}\sum_{\ell=1}^{+\infty} \sin\!\left(\frac{\pi}{L}\ell q_\text{f}\right)
                                        \sin\!\left(\frac{\pi}{L}\ell q_\text{i}\right)
              \exp\!\left(-\text{i}\frac{\hbar\pi^2\ell^2}{2mL^2}t\right)
 \end{aligned}
\end{equation}
which precisely agrees with the representation (\ref{eq:propagator2}) of the propagator.

\end{document}
