\documentclass[t,dvipsnames]{beamer}
\usepackage{mathtools}
\usepackage{siunitx}
\usepackage[T1]{fontenc}
\usepackage[sfdefault,lining,scaled=.85]{FiraSans}
\usepackage[scaled=0.85]{FiraMono}
\usepackage{newtxsf}

% https://tex.stackexchange.com/questions/435376
\makeatletter
\re@DeclareMathSymbol{\br@cext}{\mathord}{largesymbolsTXA}{32}
\makeatother

\usepackage{fontawesome}
\usepackage{pifont}
\usepackage[utf8]{inputenc}
\usepackage{mdwlist}
\usepackage{hyperref}

\usetheme{UAmnf}

\setbeamerfont{frametitle}{family=\sffamily\firamedium}
\setbeamercolor{frametitle}{fg=white}
\setbeamertemplate{navigation symbols}{}
\setbeamertemplate{headline}{  
  \leavevmode
   \begin{beamercolorbox}{logo in headline}
	\includegraphics[width=\paperwidth]{UA_mnf_headline}
   \end{beamercolorbox}
}
\setbeamercolor{alerted text}{fg=red!70!black}
\renewcommand\mathfamilydefault{\rmdefault}

\hypersetup{%
  pdftitle={Introduction to Feynman path integrals}
  ,pdfauthor={Gert-Ludwig Ingold}
  ,pdfsubject={Presentation at the Bad Honnef Physics School on
	       Methods of Path Integration in Modern Physics, Bad Honnef, 25.8.2019}
  ,pdfkeywords={Feynman path integrals, propagator, particle in a box,
                particle on a ring, harmonic oscillator}
}

\graphicspath{{./img/}}

\title[Introduction to Feynman path integrals]%
      {Introduction to Feynman path integrals}

\begin{document}

\begin{frame}[t]{}
 \vspace{2.0truecm}
 \begin{center}
   \structure{\LARGE\firamedium Introduction to Feynman path integrals}\\[0.2truecm]

   \vspace{0.1truecm}
   {\Large Gert-Ludwig Ingold}

   {\large Universität Augsburg}

   \vspace{2.3truecm}
   \faicon{github} \texttt{\normalsize github.com/gertingold/feynman-intro}

   \vspace{0.5truecm}
   {\scriptsize Bad Honnef Physics School on \textit{Methods of Path Integration in
    Modern Physics}, 25.--31.8.2019}
 \end{center}
\end{frame}

\navtrue

\section{Motivation}

\begin{frame}[c]{}
 \begin{center}
  \begin{minipage}{0.8\textwidth}
   \only<1>{\tableofcontents}%
   \only<2>{\tableofcontents[currentsection]}%
  \end{minipage}
 \end{center}
\end{frame}

\begin{frame}[t]{Classical mechanics: Hamilton vs. Lagrange}

 \vspace{-0.5truecm}
 \begin{columns}
  \begin{column}[t]{0.5\textwidth}
   \begin{center}
    \textbf{Hamilton formalism}

    Hamiltonian $H(q, p)$

    \vspace{0.2truecm}
    Poisson bracket $\{q, p\} = 1$

    equations of motion $\dot q = \{q, H\}, \dot p = \{p, H\}$
   \end{center}
  \end{column}%
  \begin{column}[t]{0.5\textwidth}
   \begin{center}
    \textbf{Lagrange formalism}

    Lagrangian $L(q, \dot q)$

    \vspace{0.2truecm}
    action $S = \int\text{d}t L(q, \dot q)$

    equation of motion from Hamilton's principle $\delta S = 0$

   \end{center}
  \end{column}%
 \end{columns}

 \begin{columns}
  \begin{column}[t]{0.5\textwidth}
   \begin{center}
    $\Arrowvert$

    canonical quantization $[\hat q, \hat p] = \text{i}\hbar$

    $\Downarrow$
   \end{center}
  \end{column}%
  \begin{column}[t]{0.5\textwidth}
   \begin{center}
    $\Arrowvert$

    ?

    $\Downarrow$
   \end{center}
  \end{column}%
 \end{columns}

 \begin{columns}
  \begin{column}[t]{0.5\textwidth}
   \begin{center}
   Heisenberg, Schrödinger, \ldots (1925)

   Schrödinger equation $\hat H\vert\Psi\rangle =
	  \text{i}\hbar\frac{\partial}{\partial t}\vert\Psi\rangle$ 
   \end{center}
  \end{column}%
  \begin{column}[t]{0.5\textwidth}
   \begin{center}
    Dirac (1933), Feynman (1948)

    \vspace{0.2truecm}
    \fcolorbox{red!70!black}{red!10}{%
       \begin{minipage}{0.75\textwidth}
	\begin{center}
         \alert{Path integral formulation of quantum mechanics}
	\end{center}
       \end{minipage}
    }
   \end{center}
  \end{column}%
 \end{columns}
\end{frame}

\begin{frame}[c]{Why base quantum mechanics\\ on Lagrange formalism?}
 path integral formulation of quantum mechanics \ldots
 \begin{itemize}
  \item \ldots can provide an alternative view on physical problems
  \item \ldots does without operators\\
	for fermionic fields, Grassmann variables are used
  \item \ldots is very well suited for relativistic field theories\\[0.2truecm]
	action $S=\int\text{d}^4x\mathcal{L}(\phi, \partial_\mu\phi)$
		 is a Minkowski scalar\\[0.2truecm]
	Hamiltonian density $\mathcal{H}(\phi, \pi)$ corresponds to the
	$0-0$ component of the energy-momentum tensor
  \item \ldots is very well suited to consider the semiclassical limit
	$S\gg\hbar$
 \end{itemize}
\end{frame}

\section{Propagator}

\begin{frame}[c]{}
 \begin{center}
  \begin{minipage}{0.8\textwidth}
   \tableofcontents[currentsection]
  \end{minipage}
 \end{center}
\end{frame}

\begin{frame}[t]{Propagator}
 Schrödinger equation\quad
 $\text{i}\hbar\dfrac{\partial}{\partial t}\vert\Psi(t)\rangle = H\vert\Psi(t)\rangle$

 \vspace{0.5truecm}
 time evolution of a state
 \begin{displaymath}
  \vert\Psi(t)\rangle = U(t)\vert\Psi(0)\rangle\quad \text{with}\quad
  U(t) = \exp\left(-\frac{\text{i}}{\hbar}Ht\right)
 \end{displaymath}
 in position representation
 \begin{displaymath}
  \textcolor{blue}{\langle q_\text{f}}\vert\Psi(t)\rangle =
   \textcolor{green!40!black}{\int\!\!\text{d}q_\text{i}}\textcolor{blue}{\langle q_\text{f}\vert}
   U(t)\textcolor{green!40!black}{\vert q_\text{i}\rangle\langle q_\text{i}}\vert\Psi(0)\rangle
 \end{displaymath}
 \begin{displaymath}
  \Psi(q_\text{f}, t) = \int\text{d}q_\text{i} K(q_\text{f}, q_\text{i}, t)\Psi(q_\text{i}, 0)
 \end{displaymath}

 \vspace{0.5truecm}
 \structure{$\Rightarrow$\quad propagator\quad
	    $K(q_\text{f}, q_\text{i}, t) = \langle q_\text{f}\vert U(t)\vert q_\text{i}\rangle$}
\end{frame}

\begin{frame}[c]{Propagator of a free particle}
 Hamiltonian of the free particle\qquad $H = \dfrac{p^2}{2m}$

 \vspace{0.5truecm}
 propagator of the free particle
 \begin{displaymath}
  \begin{aligned}
  K(q_\text{f}, q_\text{i}, t) &= \langle q_\text{f}\vert\text{e}^{-\frac{\text{i}}{\hbar}Ht}
                                   \vert q_\text{i}\rangle\\
     &= \int_{-\infty}^{+\infty}\text{d}p\langle q_\text{f}\vert p\rangle\langle p\vert q_\text{i}\rangle
	  \text{e}^{-\frac{\text{i}}{\hbar}\frac{p^2}{2m}t}\\
     &= \frac{1}{2\pi\hbar}\int_{-\infty}^{+\infty}\text{d}p
          \exp\left[-\dfrac{\text{i}}{\hbar}\left(\dfrac{p^2}{2m}t-(q_\text{f}-q_\text{i})p\right)\right]\\
     &= \frac{1}{2\pi\hbar}\exp\left(\frac{\text{i}}{\hbar}\frac{m(q_\text{f}-q_\text{i})^2}{2t}\right)
	  \int_{-\infty}^{+\infty}\text{d}p\exp\left(-\frac{\text{i}}{\hbar}\frac{t}{2m}p^2\right)
  \end{aligned}
 \end{displaymath}
\end{frame}

\begin{frame}[c]{Digression: Fresnel integral}
 \begin{columns}
  \begin{column}[b]{0.5\textwidth}
   \begin{displaymath}
    \begin{aligned}
     \int_0^\infty\text{d}x\text{e}^{\text{i}\alpha x^2}
          &= \text{e}^{\text{i}\pi/4}\!\int_0^\infty\!\text{d}u\text{e}^{-\alpha u^2}\\
          &= \frac{1}{2}\sqrt{\frac{\text{i}\pi}{\alpha}}
    \end{aligned}
   \end{displaymath}

   \begin{displaymath}
    \Rightarrow\int_{-\infty}^{+\infty}\text{d}x\text{e}^{-\text{i}\alpha x^2}
          = \sqrt{\frac{\pi}{\text{i}\alpha}}
   \end{displaymath}

  \end{column}%
  \begin{column}[b]{0.5\textwidth}
   \includegraphics[width=\textwidth]{fresnel}

   \vspace{-0.4truecm}
  \end{column}
 \end{columns}
\end{frame}

\begin{frame}[c]{Propagator of a free particle and\\ relation to classical properties}
 propagator 
 \begin{displaymath}
  K(q_\text{f}, q_\text{i}, t) = \langle q_\text{f}\vert\text{e}^{-\frac{\text{i}}{\hbar}Ht}
                                   \vert q_\text{i}\rangle
     = \sqrt{\frac{m}{2\pi\text{i}\hbar t}}\exp\left(\frac{\text{i}}{\hbar}
	    \frac{m(q_\text{f}-q_\text{i})^2}{2t}\right)
 \end{displaymath}

 \vspace{0.3truecm}
 classical motion from position $q_\text{i}$ to $q_\text{f}$ in time $t$:
 \begin{itemize}
  \item classical path\quad $q(s) = q_\text{i} + (q_\text{f}-q_\text{i})\dfrac{s}{t}$
  \item classical action\quad
	$S_\text{cl} = \int_0^t\text{d}s\dfrac{m}{2}\dot q^2
		     = \dfrac{m(q_\text{f}-q_\text{i})^2}{2t}$
 \end{itemize}

 \vspace{0.4truecm}
 propagator of free particle in terms of classical properties

 \begin{center}
  \fcolorbox{red!70!black}{red!10}{\alert{%
  $K(q_\text{f}, q_\text{i}, t) = \sqrt{-\dfrac{1}{2\pi\text{i}\hbar}
	 \dfrac{\partial^2 S_{cl}}{\partial q_\text{i}\partial q_\text{f}}}
	 \exp\left(\dfrac{\text{i}}{\hbar}S_\text{cl}\right)$
  }}
 \end{center}
\end{frame}

\begin{frame}[c]{Combining two propagators}
 semigroup property
 \begin{displaymath}
  \begin{aligned}
   K(q_\text{f}, q_\text{i}, t_2+t_1) &= \langle q_\text{f}\vert
	  \text{e}^{-\frac{\text{i}}{\hbar}Ht_2}
	  \text{e}^{-\frac{\text{i}}{\hbar}Ht_1}\vert q_\text{i}\rangle\\
     &= \textcolor{blue}{\int_{-\infty}^{+\infty}\text{d}q'}\langle q_\text{f}\vert
	  \text{e}^{-\frac{\text{i}}{\hbar}Ht_2}\textcolor{blue}{\vert q'\rangle\langle q'\vert}
	  \text{e}^{-\frac{\text{i}}{\hbar}Ht_1}\vert q_\text{i}\rangle\\
     &= \int_{-\infty}^{+\infty}\text{d}q' K(q_\text{f}, q', t_2)K(q', q_\text{i}, t_1)
  \end{aligned}
 \end{displaymath}

 \begin{columns}
  \begin{column}[b]{0.4\textwidth}
   \includegraphics[width=\textwidth]{semigroup}
  \end{column}%
  \begin{column}[b]{0.6\textwidth}
   \begin{center}
   \fbox{%
    \begin{minipage}{0.8\textwidth}
     \textit{Exercise 1:} Check the semigroup property for the propagator of a free particle.
    \end{minipage}}
   \end{center}

   \vspace{0.8truecm}
  \end{column}
 \end{columns}
\end{frame}

\section{Derivation of path integral}

\begin{frame}[c]{}
 \begin{center}
  \begin{minipage}{0.8\textwidth}
   \tableofcontents[currentsection]
  \end{minipage}
 \end{center}
\end{frame}

\begin{frame}[c]{Strategy for the derivation of the\\
	         Feynman path integral}
 basic steps:
 \begin{itemize}
  \item use semigroup property to split the propagator into
	a large number of short-time propagators
  \item use the Baker-Campbell-Hausdorff formula to separate
	kinetic and potential contributions to the Hamiltonian
        in the time-evolution operator
 \end{itemize}
\end{frame}

\begin{frame}[c]{Baker-Campbell-Hausdorff formula\\
	         and Lie-Trotter formula}
 application to a short-time propagator with operators $T$ and
 $V$ representing kinetic and potential energy, respectively
 \begin{displaymath}
   \exp\!\left(-\frac{\text{i}}{\hbar}(T+V)\Delta t\right)
	  = \exp\!\left(-\frac{\text{i}}{\hbar}V\Delta t\right)
	     \exp\!\left(-\frac{\text{i}}{\hbar}T\Delta t\right)
	     \exp\!\left(-\frac{\text{i}}{\hbar^2}X\Delta t^2\right)
 \end{displaymath}
 \begin{displaymath}
  X = \frac{\text{i}}{2}[V,T]
      -\frac{1}{6\hbar}\bigg([V,[V,T]]-2[T,[T,V]]\bigg)\Delta t+\ldots
 \end{displaymath}

 \vspace{0.3truecm}
 for $\Delta t\to 0$, the last factor may be neglected

 \structure{$\Rightarrow$ Lie-Trotter formula}
 \begin{displaymath}
  \exp\!\left(-\frac{\text{i}}{\hbar}(T+V)t\right) =
    \lim_{N\to\infty}\left[\exp\!\left(-\frac{\text{i}}{\hbar}V\frac{t}{N}\right)
      \exp\!\left(-\frac{\text{i}}{\hbar}T\frac{t}{N}\right)\right]^N
 \end{displaymath}
\end{frame}

\begin{frame}[c]{Splitting the propagator I}
 propagator\qquad $\Delta t=\dfrac{t}{N}\quad q_0=q_\text{i}\quad q_N=q_\text{f}$
 \begin{multline*}
  \hspace{-0.5truecm}\langle q_\text{f}\vert U(t)\vert q_\text{i}\rangle =\\
  \int_{-\infty}^{+\infty}\left(\prod_{n=1}^{N-1}\text{d}q_n\right)
    \langle q_\text{f}\vert \cdots\vert q_n\rangle\underbrace{\langle q_n\vert
	 \text{e}^{-\frac{\text{i}}{\hbar}V\Delta t}
	 \text{e}^{-\frac{\text{i}}{\hbar}T\Delta t}
	 \vert q_{n-1}\rangle}_{\mathclap{\text{\small $%
	 \sqrt{\frac{m}{2\pi\text{i}\hbar\Delta t}}\exp\left[\frac{\text{i}}{\hbar}
	 \left(\frac{m(q_n-q_{n-1})^2}{2\Delta t}-V(q_n)\Delta t\right)\right]$}}}
	 \langle q_{n-1}\vert\cdots\vert q_\text{i}\rangle
 \end{multline*}
 \begin{center}
  \includegraphics[width=0.6\textwidth]{splitting}
 \end{center}
\end{frame}

\begin{frame}[c]{Splitting the propagator II}
 \begin{multline*}
  \langle q_\text{f}\vert U(t)\vert q_\text{i}\rangle
  = \sqrt{\frac{m}{2\pi\text{i}\hbar\Delta t}}\int_{-\infty}^{+\infty}
    \prod_{n=1}^{N-1}\left(\sqrt{\frac{m}{2\pi\text{i}\hbar\Delta t}}\text{d}q_n\right)\\
    \times\exp\left[\frac{\text{i}}{\hbar}\sum_{n=1}^N\left(
	 \frac{m}{2}\bigg(\frac{q_n-q_{n-1}}{\Delta t}\bigg)^2
	 -V(q_n)\right)\Delta t\right]
 \end{multline*}
 \begin{center}
  \includegraphics[width=0.6\textwidth]{splitting}
 \end{center}

 \vspace{0.5truecm}
 \begin{center}
  let us now take the continuum limit $N\to\infty$
 \end{center}
\end{frame}

\begin{frame}[c]{Action}
 \begin{displaymath}
  \lim_{N\to\infty}\sum_{n=1}^N\left(\frac{m}{2}\bigg(\frac{q_n-q_{n-1}}{\Delta t}\bigg)^2
                                     -V(q_n)\right)\Delta t
	= \int_0^t\text{d}s\left(\frac{m}{2}\dot q(s)^2-V(q)\right)
 \end{displaymath}

 in the exponent of the propagator, we find the action functional
 \begin{displaymath}
  S[q] = \int_0^t\text{d}s\left(\frac{m}{2}\dot q(s)^2-V\big(q(s)\big)\right)
 \end{displaymath}
 which takes a function $q(s)$ and returns a number $S$

 \begin{center}
  \structure{%
  $\Rightarrow$\quad%
  \begin{minipage}[t]{0.7\textwidth}
   \raggedright we have obtained a connection between the Lagrange formalism
   of classical mechanics and the quantum mechanical propagator
  \end{minipage}}
 \end{center}
\end{frame}

\begin{frame}[c]{Feynman path integral}
 propagator as functional integral (Feynman path integral)

 \begin{center}
  \fcolorbox{red!70!black}{red!10}{\alert{%
    $\displaystyle K(q_\text{f}, q_\text{i}, t) = \int_{q(0)=q_\text{i}}^{q(t)=q_\text{f}}\mathcal{D}q\,
    exp\left(\frac{\text{i}}{\hbar}S[q]\right)$
  }}
 \end{center}

 \vspace{0.5truecm}
 the integral runs over all trajectories $q(s)$ satisfying the boundary
 conditions $q(0)=q_\text{i}$ and $q(t)=q_\text{f}$

 \begin{center}
  \includegraphics[width=0.8\textwidth]{feynman}
 \end{center}
\end{frame}

\begin{frame}[c]{Evaluation of a Feynman path integral}
 \begin{itemize}
  \item evaluate discretized version as an $N$-dimensional integral\\
	for linear systems, the following relation is useful:
	\begin{displaymath}
	 \int_{-\infty}^{+\infty}\text{d}x^N \exp\!\left(-\mathbf{x}^\text{T}
	    \mathbf{A}\hspace{0.1em}\mathbf{x}\right) =
		\sqrt{\frac{\pi^N}{\det(\mathbf{A})}}
	\end{displaymath}
  \item expansion in a complete set of functions\\
	$\rightarrow$ see our later discussion of the harmonic oscillator
 \end{itemize}

 \vspace{0.5truecm}
 \begin{center}
 \fbox{%
  \begin{minipage}{0.75\textwidth}
   \textit{Exercise 2:} Evaluate the Feynman path integral for the free particle
   by means of discretization. Why is the result independent of $N$?
  \end{minipage}}
 \end{center}
\end{frame}

\begin{frame}[t]{Physical interpretation}
 \begin{center}
  \includegraphics[width=0.6\textwidth]{feynman}
 \end{center}

 \begin{itemize}
  \item All trajectories connecting initial to final point in the given
	time contribute to the propagator.
  \item Almost all of these trajectories are not solutions of the
        corresponding classical equation of motion.
  \item Each trajectory contributes with a phase factor depending on 
	the action associated with the trajectory, cf. double slit.
  \item The nonclassical trajectories can be viewed as contributions
	of quantum fluctuations.
 \end{itemize}
\end{frame}

\begin{frame}[t]{Classical trajectory and\\ quantum fluctuations}
 \vspace{0.4truecm}
 Conventional integral as an analogy:

 \vspace{0.5truecm}
 \begin{columns}
  \begin{column}[b]{0.37\textwidth}
   $S(x)=x^2$
	  
   $x=0$ corresponds to\\[-0.1truecm] classical trajectory

   \vspace{0.8truecm}
   $f(x)=\cos(S(x))$

   \vspace{0.9truecm}
   $I(x)=\int_{-x}^x\text{d}u\cos(S(u))$

   \vspace{0.3truecm}
  \end{column}\hspace{-0.5truecm}%
  \begin{column}[b]{0.6\textwidth}
   \includegraphics[width=\textwidth]{saddlepoint}
  \end{column}%
 \end{columns}

 The dominant contributions come from stationary points of the action and
 fluctuations around it.
\end{frame}

\section{Particle on a ring}

\begin{frame}[c]{}
 \begin{center}
  \begin{minipage}{0.8\textwidth}
   \tableofcontents[currentsection]
  \end{minipage}
 \end{center}
\end{frame}

\begin{frame}[t]{Standard approach}
 particle of mass $m$ on a ring of radius $R$

 \vspace{0.3truecm}
 Hamiltonian\quad $H = -\dfrac{\hbar^2}{2mR^2}\dfrac{\text{d}^2}{\text{d}\phi^2}$

 \vspace{0.5truecm}
 eigenfunctions with $\psi(0) = \psi(2\pi)$ and $\psi'(0) = \psi'(2\pi)$
 \begin{displaymath}
  \psi_\ell(\phi) = \frac{1}{\sqrt{2\pi}}\text{e}^{\text{i}\ell\phi}\qquad
  \ell = 0, \pm1, \pm2, \ldots
 \end{displaymath}

 eigenenergies
 \begin{displaymath}
  E_\ell = \frac{\hbar^2\ell^2}{2mR^2}
 \end{displaymath}

 propagator
 \begin{displaymath}
  K(\phi_\text{f}, \phi_\text{i}, t) = \frac{1}{2\pi}\sum_{\ell=-\infty}^{+\infty}
	 \exp\left(\text{i}\ell(\phi_\text{f}-\phi_\text{i})
		   -\text{i}\frac{\hbar\ell^2}{2mR^2}t\right)
 \end{displaymath}
\end{frame}

\begin{frame}[t]{More than one way}
 two ways to reach $\phi_\text{f}$:\qquad
	\raisebox{-2truecm}{\includegraphics[width=0.6\textwidth]{ring_1}}

 \vspace{0.5truecm}
 \begin{itemize}
  \item the two paths cannot be deformed into each other
 \end{itemize}
 \begin{center}
  \includegraphics[width=0.53\textwidth]{ring_deform}
 \end{center}
\end{frame}

\begin{frame}[t]{More than one way}
 two ways to reach $\phi_\text{f}$:\qquad
	\raisebox{-2truecm}{\includegraphics[width=0.6\textwidth]{ring_1}}

 \vspace{0.5truecm}
 \begin{itemize}
  \item the two paths cannot be deformed into each other
  \item two contributions:\\[0.2truecm]
	$R\sqrt{\dfrac{m}{2\pi\text{i}\hbar t}}\exp\left(\dfrac{\text{i}}{\hbar}
		 \dfrac{mR^2}{2}\dfrac{(\phi_\text{f}-\phi_\text{i})^2}{t}\right)$
	\quad and\\[0.2truecm]
	$R\sqrt{\dfrac{m}{2\pi\text{i}\hbar t}}\exp\left(\dfrac{\text{i}}{\hbar}
        \dfrac{mR^2}{2}\dfrac{\big((\phi_\text{f}-2\pi)-\phi_\text{i}\big)^2}{t}\right)$
 \end{itemize}
 \begin{flushright}
  but there is more \ldots
 \end{flushright}
\end{frame}

\begin{frame}[t]{Winding numbers and propagator}
 \begin{center}
  \includegraphics[width=0.8\textwidth]{ring_2}
 \end{center}

 \begin{itemize}
  \item the angles $\phi_\text{f}$ and $\phi_\text{f}+2\pi n$ have
	to be identfied
  \item the propagator is a sum over all topologically distinct contributions
	\begin{displaymath}
	 K(\phi_\text{f}, \phi_\text{i}, t) = R\sqrt{\dfrac{m}{2\pi\text{i}\hbar t}}
         \sum_{n=-\infty}^{+\infty}\exp\left(\dfrac{\text{i}}{\hbar}\dfrac{mR^2}{2}
 	 \dfrac{(\phi_\text{f}-\phi_\text{i}-2\pi n)^2}{t}\right)
	\end{displaymath}
 \end{itemize}
\end{frame}

\begin{frame}[c]{Dual representations}
 two different expressions for the propagator

 \vspace{0.4truecm}
 \structure{winding number}
 \begin{displaymath}
  K(\phi_\text{f}, \phi_\text{i}, t) = R\sqrt{\dfrac{m}{2\pi\text{i}\hbar t}}
      \sum_{n=-\infty}^{+\infty}\exp\left(\dfrac{\text{i}}{\hbar}\dfrac{mR^2}{2}
      \dfrac{(\phi_\text{f}-\phi_\text{i}-2\pi n)^2}{t}\right)
 \end{displaymath}

 \vspace{0.2truecm}
 \structure{angular momentum}
 \begin{displaymath}
  K(\phi_\text{f}, \phi_\text{i}, t) = \frac{1}{2\pi}\sum_{\ell=-\infty}^{+\infty}
	 \exp\left(\text{i}\ell(\phi_\text{f}-\phi_\text{i})
		   -\text{i}\frac{\hbar\ell^2}{2mR^2}t\right)
 \end{displaymath}

 \vspace{0.2truecm}
 compare with propagator of free particle
 \begin{displaymath}
  K(q_\text{f}, q_\text{i}, t)
    = \frac{1}{2\pi}\int_{-\infty}^{+\infty}\frac{\text{d}p}{\hbar}
	  \exp\left(\text{i}\frac{p}{\hbar}(q_\text{f}-q_\text{i})
		    -\text{i}\frac{p^2}{2\hbar m}t\right)
 \end{displaymath}
\end{frame}

\begin{frame}[c]{A useful relation}
 Fourier representation of a periodic $\delta$-comb
 \begin{center}
  \includegraphics[width=0.6\textwidth]{deltacomb}
 \end{center}

 \vspace{0.2truecm}
 \begin{displaymath}
  \sum_{n=-\infty}^{+\infty}\delta(x-2\pi n)
  = \frac{1}{2\pi}\sum_{\ell=-\infty}^{+\infty}\text{e}^{\text{i}\ell x}
 \end{displaymath}

 \vspace{0.4truecm}
 \begin{center}
 \fbox{%
  \begin{minipage}{0.43\textwidth}
   \textit{Exercise 3:} Verify this relation.
  \end{minipage}}
 \end{center}
\end{frame}

\begin{frame}[c]{Relating the two representations}
 \begin{footnotesize}
  \begin{displaymath}
   \begin{aligned}
    K(\phi_\text{f}, \phi_\text{i}, t)
    &= R\sqrt{\frac{m}{2\pi\text{i}\hbar t}}
        \sum_{n=-\infty}^{+\infty}\exp\!\left(\frac{\text{i}}{\hbar}\frac{mR^2}{2}
        \frac{(\phi_\text{f}-\phi_\text{i}-2\pi n)^2}{t}\right)\\
    &= R\sqrt{\frac{m}{2\pi\text{i}\hbar t}}\int_{-\infty}^{+\infty}\!\!\text{d}\phi
       \exp\!\left(\frac{\text{i}}{\hbar}\frac{mR^2}{2t}\phi^2\right)
       \sum_{n=-\infty}^{+\infty}\delta\big(\phi-(\phi_\text{f}-\phi_\text{i}-2\pi n)\big)\\
    &= \frac{R}{2\pi}\sqrt{\frac{m}{2\pi\text{i}\hbar t}}\sum_{\ell=-\infty}^{+\infty}
       \exp\!\big(-\text{i}\ell(\phi_\text{f}-\phi_\text{i})\big)\int_{-\infty}^{+\infty}\!\!\text{d}\phi
       \exp\!\left(\frac{\text{i}}{\hbar}\frac{mR^2}{2t}\phi^2+\text{i}\ell\phi\right)\\
    &= \frac{1}{2\pi}\sum_{\ell=-\infty}^{+\infty}\exp\!\left(\text{i}\ell(\phi_\text{f}-\phi_\text{i})
	   -\text{i}\frac{\hbar\ell^2}{2mR^2}t\right)\qquad\text{\normalsize\ding{51}}
   \end{aligned}
  \end{displaymath}
 \end{footnotesize}

 \vspace{0.4truecm}
 \begin{center}
 \fbox{%
  \begin{minipage}{0.53\textwidth}
   \textit{Exercise 4:} Fill in the missing details.
  \end{minipage}}
 \end{center}
\end{frame}

\section{Particle in a box}

\begin{frame}[c]{}
 \begin{center}
  \begin{minipage}{0.8\textwidth}
   \tableofcontents[currentsection]
  \end{minipage}
 \end{center}
\end{frame}

\begin{frame}[t]{Standard approach}
 particle of mass $m$ in an infinitely deep potential well of width $L$

 \vspace{0.3truecm}
 Hamiltonian\quad $H = -\dfrac{\hbar^2}{2m}\dfrac{\text{d}^2}{\text{d}q^2}$

 \vspace{0.5truecm}
 eigenfunctions with $\psi(0) = \psi(L) = 0$
 \begin{displaymath}
  \psi_j(q) = \sqrt{\frac{2}{L}}\sin\!\left(\frac{\pi j}{L}q\right)\qquad
  j = 1, 2, 3, \ldots
 \end{displaymath}

 eigenenergies
 \begin{displaymath}
  E_j = \frac{\hbar^2\pi^2}{2mL^2}j^2
 \end{displaymath}

 propagator
 \begin{displaymath}
  K(q_\text{f}, q_\text{i}, t) = \frac{2}{L}\sum_{j=1}^{\infty}
    \exp\left(-\text{i}\frac{\hbar\pi^2j^2}{2mL^2}t\right)
	 \sin\!\left(\frac{\pi j}{L}q_\text{f}\right)
	 \sin\!\left(\frac{\pi j}{L}q_\text{i}\right)
 \end{displaymath}
\end{frame}

\begin{frame}[t]{Possible trajectories}

 \vspace{-0.5truecm}
 \begin{center}
  \includegraphics[width=0.5\textwidth]{traj_box}
 \end{center}

 unfolding of the trajectories
 \begin{center}
  \includegraphics[width=0.6\textwidth]{unfolding}
 \end{center}

 \begin{itemize}
  \item sum over propagators of a free particle
  \item relative phase between the various contributions?
 \end{itemize}
\end{frame}

\begin{frame}[t]{Influence of a wall}
 \begin{center}
  \includegraphics[width=0.6\textwidth]{wall}
 \end{center}

 subtract the contribution of paths crossing the wall:
 \begin{displaymath}
  K_\text{wall}(q_\text{f}, q_\text{i}, t)
  = K_\text{free}(q_\text{f}, q_\text{i}, t) - K_\text{free}(-q_\text{f}, q_\text{i}, t)
 \end{displaymath}

 \begin{itemize}
  \item compare with method of images in electrostatics
  \item each reflection yields a factor $-1$
 \end{itemize}

 \begin{footnotesize}
  \begin{flushright}
   A. Auerbach, L. S. Schulman, J. Phys. A: Math. Gen. \textbf{30}, 5993 (1997)
  \end{flushright}
 \end{footnotesize}
\end{frame}

\begin{frame}[t]{Propagator for particle in a box}

 \vspace{-0.3truecm}
 \begin{center}
  \includegraphics[width=0.7\textwidth]{images}
 \end{center}

 \textcolor{red!60!black}{even number of reflections at $q = q_\text{f}+2nL$}\\
 \textcolor{blue!60!black}{odd number of reflections at $q = 2nL-q_\text{f}$}

 \vspace{0.1truecm}
 propagator
 \begin{displaymath}
  \begin{aligned}
   K(q_\text{f}, q_\text{i}, t) &= \sqrt{\frac{m}{2\pi\text{i}\hbar t}}\sum_{n=-\infty}^{+\infty}
    \left[\exp\!\left(\frac{\text{i}}{\hbar}
	  \frac{m(2nL+q_\text{f}-q_\text{i})^2}{2t}\right)\right.\\
    &\hspace{3truecm}-\left.\exp\!\left(\frac{\text{i}}{\hbar}
	  \frac{m(2nL-q_\text{f}-q_\text{i})^2}{2t}\right)\right]
  \end{aligned}
 \end{displaymath}

 \begin{scriptsize}
  \begin{flushright}
   W. Janke, H. Kleinert, Lett. Nuovo Cim. \textbf{25}, 297 (1979);
   M. Goodman, Am. J. Phys. \textbf{49}, 843 (1981)
  \end{flushright}
 \end{scriptsize}
\end{frame}

\begin{frame}[t]{Two versions of the propagator}
 again, we have obtained two dual versions of the propagator:
 \begin{displaymath}
  K(q_\text{f}, q_\text{i}, t) = \frac{2}{L}\sum_{j=1}^{\infty}
    \exp\left(-\text{i}\frac{\hbar\pi^2j^2}{2mL^2}t\right)
	 \sin\!\left(\frac{\pi j}{L}q_\text{f}\right)
	 \sin\!\left(\frac{\pi j}{L}q_\text{i}\right)
 \end{displaymath}
 and
 \begin{displaymath}
  \begin{aligned}
   K(q_\text{f}, q_\text{i}, t) &= \sqrt{\frac{m}{2\pi\text{i}\hbar t}}\sum_{n=-\infty}^{+\infty}
    \left[\exp\!\left(\frac{\text{i}}{\hbar}
	  \frac{m(2nL+q_\text{f}-q_\text{i})^2}{2t}\right)\right.\\
    &\hspace{3truecm}-\left.\exp\!\left(\frac{\text{i}}{\hbar}
	  \frac{m(2nL-q_\text{f}-q_\text{i})^2}{2t}\right)\right]
  \end{aligned}
 \end{displaymath}

 \vspace{0.2truecm}
 \begin{center}
 \fbox{%
  \begin{minipage}{0.92\textwidth}
   \textit{Exercise 5:} Show that the two representations of the propagator
	  of a particle in a box agree. Follow the strategy which proved
	  successful already for the particle on a ring.
  \end{minipage}}
 \end{center}
\end{frame}

\section{Harmonic oscillator}

\begin{frame}[c]{}
 \begin{center}
  \begin{minipage}{0.8\textwidth}
   \tableofcontents[currentsection]
  \end{minipage}
 \end{center}
\end{frame}

\begin{frame}[c]{Path integral for the harmonic oscillator}
 \structure{path integral representation of the propagator of the\\ harmonic oscillator}
 \begin{displaymath}
  K(q_\text{f}, q_\text{i}, t) = \int_{q(0)=q_\text{i}}^{q(t)=q_\text{f}}
    \mathcal{D}q\exp\left[\frac{\text{i}}{\hbar}\int_0^t\text{d}s
	 \frac{m}{2}\left(\left(\frac{\text{d}q}{ds}\right)^2-\omega^2q^2\right)\right]
 \end{displaymath}

 \vspace{0.5truecm}
 How do we integrate over all functions satisfying the boundary conditions
 $q(0)=q_\text{i}$ and $q(t)=q_\text{f}$?

 \begin{itemize}
  \item decompose path into two contributions:
	\begin{itemize}
         \item[--] a path satisfying the boundary conditions
         \item[--] fluctuations around this path described by a complete
		   set of functions vanishing at the boundaries
	\end{itemize}
  \item integrate over the expansion coefficients of the fluctuations
 \end{itemize}
\end{frame}

\begin{frame}[t]{Path decomposition and action}
 \vspace{-0.3truecm}
 \begin{displaymath}
  q(s) = \bar q(s) + \xi(s)
 \end{displaymath}
 \begin{displaymath}
  \text{with}\quad q(0)=q_\text{i}, q(t)=q_\text{f}\quad\text{and}\quad \xi(0)=\xi(t)=0
 \end{displaymath}
 \structure{action}
 \begin{small}
  \begin{align}
   S[q] &= \int_0^t\text{d}s\frac{m}{2}\left[\left(\frac{\text{d}q}{\text{d}s}\right)^2
 	   -\omega^2q^2\right]\nonumber\\
        &= \int_0^t\text{d}s\frac{m}{2}\left[\left(\frac{\text{d}\bar q}{\text{d}s}\right)^2
           -\omega^2\bar q^2+2\frac{\text{d}\bar q}{\text{d}s}\frac{\text{d}\xi}{\text{d}s}
           -2\omega^2\bar q\xi+\left(\frac{\text{d}\xi}{\text{d}s}\right)^2-\omega^2\xi^2\right]
	   \nonumber\\
   \intertext{integration by parts}
        &= \int_0^t\text{d}s\frac{m}{2}\left[\left(\frac{\text{d}\bar q}{\text{d}s}\right)^2
           -\omega^2\bar q^2-2\left(\frac{\text{d}^2\bar q}{\text{d}s^2}+\omega^2\bar q\right)\xi
           +\left(\frac{\text{d}\xi}{\text{d}s}\right)^2-\omega^2\xi^2\right]\nonumber
  \end{align}
 \end{small}
 linear term in $\xi$ vanishes if $\bar q(s)$ solves the equation of motion
\end{frame}

\begin{frame}[t]{Expansion around classical solution}
 \begin{displaymath}
  q(s) = q_\text{cl}(s)+\xi(s)
 \end{displaymath}

 \structure{action}
 \begin{displaymath}
  \begin{aligned}
   S[q] &= \int_0^t\text{d}s\frac{m}{2}
	   \left[\left(\frac{\text{d}q_\text{cl}}{\text{d}s}\right)^2-\omega^2q_\text{cl}^2
	   +\left(\frac{\text{d}\xi}{\text{d}s}\right)^2-\omega^2\xi^2\right]\\
        &= S_\text{cl} + \int_0^t\text{d}s\frac{m}{2}\left[
	   \left(\frac{\text{d}\xi}{\text{d}s}\right)^2-\omega^2\xi^2\right]\\
  \end{aligned}
 \end{displaymath}

 \begin{itemize}
  \item expansion around the classical path makes linear term in $\xi$ disappear because
	$\delta S=0$ for classical paths
  \item for the harmonic oscillator:
	\begin{itemize}
	 \item classical action $S_\text{cl}$ contains the full dependence on $q_\text{i}$
	       and $q_\text{f}$
	 \item fluctuation part is independent of $q_\text{i}$ and $q_\text{f}$ and only
	       contributes to the prefactor
	\end{itemize}
 \end{itemize}
\end{frame}

\begin{frame}[t]{Classical solution}
 general solution of the equation of motion of the \structure{harmonic oscillator}
 \begin{displaymath}
  q_\text{cl}(s) = A\cos(\omega s) + B\sin(\omega s)
 \end{displaymath}
 boundary conditions
 \begin{displaymath}
  \begin{aligned}
   q_\text{cl}(0) &= q_\text{i} = A\\
   q_\text{cl}(t) &= q_\text{f} = A\cos(\omega t)+B\sin(\omega t)
  \end{aligned}
 \end{displaymath}
 classical solution satisfying the boundary conditions
 \begin{displaymath}
  \begin{aligned}
   q_\text{cl}(s) &= q_\text{i}\cos(\omega s)
		     + \frac{q_\text{f}-q_\text{i}\cos(\omega t)}{\sin(\omega t)}\sin(\omega s)\\
		  &= q_\text{i}\frac{\sin\big(\omega(t-s)\big)}{\sin(\omega t)}
		     + q_\text{f}\frac{\sin(\omega s)}{\sin(\omega t)}
  \end{aligned}
 \end{displaymath}
\end{frame}

\begin{frame}[t]{Existence of classical solution}
 \begin{displaymath}
   q_\text{cl}(s) = q_\text{i}\frac{\sin\big(\omega(t-s)\big)}{\sin(\omega t)}
		    + q_\text{f}\frac{\sin(\omega s)}{\sin(\omega t)}
 \end{displaymath}

 \vspace{0.3truecm}
 What happens for $t=\dfrac{\pi n}{\omega}$?

 \vspace{0.3truecm}
 boundary value problems \ldots
 \begin{itemize}
  \item \ldots do not necessarily have a solution
  \item \ldots may have more than one solution
 \end{itemize}

 \begin{center}
  \includegraphics[width=0.7\textwidth]{ho_classical}
 \end{center}
\end{frame}

\begin{frame}[t]{Classical action}
 \begin{align}
  S_\text{cl} &= \int_0^t\text{d}s\frac{m}{2}\left(\dot q_\text{cl}^2(s)
	            -\omega^2q_\text{cl}^2(s)\right)\nonumber\\
  \intertext{integration by parts}
	      &= \left.\frac{m}{2}q_\text{cl}(s)\dot q_\text{cl}(s)\right\vert_0^t
		 -\int_0^t\text{d}s\frac{m}{2}q_\text{cl}(s)\left(\ddot q_\text{cl}(s)
			  +\omega^2q_\text{cl}(s)\right)\nonumber\\
	      &=\frac{m}{2}\big(q_\text{cl}(t)\dot q_\text{cl}(t)
	                 -q_\text{cl}(0)\dot q_\text{cl}(0)\big)\nonumber
 \end{align}
 \begin{displaymath}
  \dot q_\text{cl}(0) = \omega\frac{q_\text{f}-q_\text{i}\cos(\omega t)}{\sin(\omega t)}\qquad
  \dot q_\text{cl}(t) = \omega\frac{q_\text{f}\cos(\omega t)-q_\text{i}}{\sin(\omega t)}
 \end{displaymath}

 \vspace{0.3truecm}
 \structure{classical action of the harmonic oscillator}
 \begin{displaymath}
  S_\text{cl} = \frac{m\omega}{2}\frac{\big(q_\text{f}^2+q_\text{i}^2\big)\cos(\omega t)
	                               -2q_\text{f}q_\text{i}}{\sin(\omega t)}
 \end{displaymath}
\end{frame}

\begin{frame}[c]{Expansion of the fluctuations}
 fluctuation part of the action
 \begin{displaymath}
  \begin{aligned}
   \int_0^t\text{d}s\frac{m}{2}\left(\dot\xi^2-\omega^2\xi^2\right)
 	&= \underbrace{\left.\frac{m}{2}\xi(s)\dot\xi(s)\right\vert_0^t}_{=0}
 	   -\int_0^t\text{d}s\frac{m}{2}\xi\left(\ddot\xi+\omega^2\xi\right)\\
	  &=-\int_0^t\text{d}s\frac{m}{2}\xi\left(\frac{\text{d}^2}{\text{d}s^2}+\omega^2\right)\xi
  \end{aligned}
 \end{displaymath}

 expand the fluctuations in the complete set of eigenfunctions of $\frac{\text{d}^2}{\text{d}s^2}+\omega^2$
 vanishing at the boundaries:
 \begin{displaymath}
  -\left(\frac{\text{d}^2}{\text{d}s^2}+\omega^2\right)\xi_n(s) = \lambda_n\xi_n(s)
 \end{displaymath}
\end{frame}

\begin{frame}[c]{Eigenfunctions and -values}
 \begin{displaymath}
  \xi_n(s) = \sqrt{\frac{2}{t}}\sin\left(\pi n\frac{s}{t}\right)\qquad
  \lambda_n = \frac{\pi^2n^2}{t^2}-\omega^2\qquad n=1, 2,\ldots
 \end{displaymath}

 \begin{itemize}
  \item $0\leq \omega t < \pi$:\quad $\lambda_n > 0$ for all $n$\\
	classical action is a minimum
  \item $\omega t=\pi$: $a_1\xi_1(s)$ is a classical solution for all amplitudes $a_1$\\
	\includegraphics[width=0.5\textwidth]{ho_classical}
  \item $\pi < \omega t < 2\pi$:\quad $\lambda_1 <0,\quad \lambda_n >0$ for $n=2, 3,\ldots$\\
	classical action is a saddle point
  \item \ldots
 \end{itemize}
\end{frame}

\begin{frame}[c]{Total action and propagator}
 \structure{action}
 \begin{displaymath}
  S[q] = S_\text{cl}+\sum_{n=1}^{\infty}\frac{m}{2}\left(\frac{\pi^2n^2}{t^2}-\omega^2\right)a_n^2
 \end{displaymath}

 \structure{propagator}
 \begin{displaymath}
  \begin{aligned}
   K(q_\text{f}, q_\text{i},t) &\sim \int_{-\infty}^{+\infty}\cdots\int_{-\infty}^{+\infty} 
     \left(\prod_{n=1}^\infty\text{d}a_n\right)\exp\left[\frac{\text{i}}{\hbar}\left(
       S_\text{cl}+\frac{m}{2}\sum_{n=1}^\infty\lambda_na_n^2\right)\right]\\
   &\sim \left(\prod_{n=1}^\infty\lambda_n\right)^{-1/2}\exp\left(\frac{\text{i}}{\hbar}S_\text{cl}\right)
  \end{aligned}
 \end{displaymath}

 \structure{fluctuation determinant}
 \begin{displaymath}
  \det\left(-\frac{\text{d}^2}{\text{d}s^2}-\omega^2\right) = \prod_{n=1}^\infty\lambda_n
 \end{displaymath}
\end{frame}

\begin{frame}[c]{Prefactor}
 the prefactor unknown so far does not depend on $\omega$\\
 $\rightarrow$ compare with propagator of the free particle
 
 \vspace{0.3truecm}
 \structure{classical action}
 \begin{displaymath}
  \lim_{\omega\to 0}\frac{m}{2}\underbrace{\frac{\omega}{\sin(\omega t)}}_{\to 1/t}
    \underbrace{\big((q_\text{f}^2+q_\text{i}^2)\cos(\omega t)-2q_\text{f}q_\text{i}\big)}_{\to
	     (q_\text{f}-q_\text{i})^2}
   = \frac{m}{2}\frac{(q_\text{f}-q_\text{i})^2}{t}\qquad\text{\ding{51}}
 \end{displaymath}

 \structure{prefactor}
 \begin{displaymath}
  \sqrt{\prod_{n=1}^\infty\frac{\lambda_n^{(0)}}{\lambda_n}}\sqrt{\frac{m}{2\pi\text{i}\hbar t}}
 \end{displaymath}

 with
 \begin{displaymath}
  \lambda_n = \frac{\pi^2n^2}{t^2}-\omega^2\qquad\text{and}\qquad
  \lambda_n^{(0)} = \frac{\pi^2n^2}{t^2}
 \end{displaymath}
\end{frame}

\begin{frame}[c]{Ratio of determinants}
 \begin{displaymath}
  \prod_{n=1}^\infty\frac{\lambda_n}{\lambda_n^{(0)}}
     = \prod_{n=1}^\infty\left(1-\big(\frac{\omega t}{\pi n}\big)^2\right)
     = \frac{\sin(\omega t)}{\omega t}
 \end{displaymath}

 \vspace{0.3truecm}
 \structure{propagator of the harmonic oscillator}
 \begin{footnotesize}
  \begin{displaymath}
   \begin{aligned}
     K(q_\text{f}, q_\text{i}, t) &= \sqrt{\frac{m\omega}{2\pi\text{i}\hbar\sin(\omega t)}}
         \exp\left(\text{i}\frac{m\omega}{2\hbar}\frac{(q_\text{f}^2+q_\text{i}^2)\cos(\omega t)
		      -2q_\text{i}q_\text{f}}{\sin(\omega t)}\right)\\
      &= \sqrt{\frac{m\omega}{2\pi\hbar\vert{\sin(\omega t)}\vert}}
         \exp\left(\text{i}\frac{m\omega}{2\hbar}\frac{(q_\text{f}^2+q_\text{i}^2)\cos(\omega t)
         -2q_\text{i}q_\text{f}}{\sin(\omega t)}-\text{i}\left(\frac{\pi}{4}+n\frac{\pi}{2}\right)\right)
   \end{aligned}
  \end{displaymath}

  $n$: number of zeros of the sine function crossed up to time $t$
 \end{footnotesize}

 \vspace{0.3truecm}
 This choice of the phase ensures the validity of the semigroup property, e.g.
 \begin{displaymath}
  K(q_\text{f}, q', \tfrac{3\pi}{2\omega}) = \int_{-\infty}^{+\infty}\text{d}q'
	     K(q_\text{f}, q', \tfrac{3\pi}{4\omega})K(q', q_\text{i}, \tfrac{3\pi}{4\omega})
 \end{displaymath}
\end{frame}

\begin{frame}[t]{Exercises for the harmonic oscillator}
 \begin{center}
 \fbox{%
  \begin{minipage}{\textwidth}
   \textit{Exercise 6:} Use the Mehler formula
   \begin{displaymath}
    \sum_{n=0}^\infty\frac{H_n(x)H_n(y)}{n!}\left(\frac{z}{2}\right)^n
      = \frac{1}{\sqrt{1-z^2}}\exp\!\left(\frac{2xyz-(x^2+y^2)z^2}{1-z^2}\right)
   \end{displaymath}
   to determine the propagator of the harmonic oscillator starting from its stationary
   wave functions and eigenenergies.
  \end{minipage}}

 \vspace{0.3truecm}
 \fbox{%
  \begin{minipage}{\textwidth}
   \textit{Exercise 7:} Check the choice of phases in the propagator of the harmonic
   oscillator by combining two propagators over time intervals $3\pi/4\omega$ to a
   propagator over the time interval $3\pi/2\omega$ by means of the semigroup property.
  \end{minipage}}
 \end{center}
\end{frame}

\begin{frame}[c]{Nonlinear potentials}
 action
 \begin{displaymath}
  S[q] = \int_0^t\text{d}s\left(\frac{m}{2}\dot q^2 - V(q)\right)
 \end{displaymath}

 \vspace{0.2truecm}
 expansion around the classical solution: $q(s) = q_\text{cl}(s)+\xi(s)$

 \begin{footnotesize}
  \begin{displaymath}
   \begin{aligned}
    S &= \int_0^t\text{d}s\left(\frac{m}{2}(\dot q_\text{cl}^2+2\dot q_\text{cl}\dot\xi+\dot\xi^2)
         -V(q_\text{cl})-V'(q_\text{cl})\xi-\frac{1}{2}V''(q_\text{cl})\xi^2+\ldots\right)\\
      &= S_\text{cl} - \int_0^t\text{d}s\underbrace{\big(m\ddot q_\text{cl}+V'(q_\text{cl})\big)}_{=0}\xi
         -\frac{1}{2}\int_0^t\text{d}s \xi\left(m\frac{\text{d}^2}{\text{d}s^2}+V''(q_\text{cl})\right)\xi
	 + \ldots
   \end{aligned}
  \end{displaymath}
 \end{footnotesize}

 \begin{itemize}
  \item the quadratic contribution in the fluctuations generally depends on $q_\text{f}$ and $q_\text{i}$
  \item it leads again to a determinant
  \item the expansion of $\xi$ does not necessarily need to be done in terms of eigenfunctions 
 \end{itemize}
\end{frame}

\begin{frame}[c]{Fluctuation amplitude}
 \begin{itemize}
  \item the second-order term in the fluctuations determines their typical amplitude
	\begin{displaymath}
         \frac{\xi^2}{\hbar} \sim 1\quad \rightarrow\quad \xi \sim \sqrt{\hbar}
	\end{displaymath}
  \item the third-order term is smaller by a factor $\sqrt{\hbar}$ and can be neglected in the
	semiclassical limit
  \item however, higher-order terms can become relevant when the eigenvalue of a fluctuation
	  mode approaches zero\\[0.1truecm]
	example: for a harmonic oscillator around $t=\pi/\omega$, weak anharmonicities of
		 the potential may become relevant
 \end{itemize}
\end{frame}

\begin{frame}[c]{Literature}
 \begin{itemize}
  \item G.-L. Ingold\\
        \textit{Path Integrals and Their Application to Dissipative Quantum Systems}\\
        Lect.  Notes Phys. \textbf{611}, 1 (2002) [\url{arXiv:quant-ph/0208026}]
  \item H. Kleinert\\
	\textit{Path Integrals in Quantum Mechanics, Statistics, Polymer Physics, and Financial Markets}\\
	(World Scientific, 2009)
 \end{itemize}
\end{frame}
\end{document}
