\documentclass[t,dvipsnames]{beamer}
\usepackage{amsmath}
\usepackage{siunitx}
\usepackage[T1]{fontenc}
\usepackage[sfdefault,lining,scaled=.85]{FiraSans}
\usepackage[scaled=0.85]{FiraMono}
\usepackage{newtxsf}
\usepackage{pifont}
\usepackage[utf8]{inputenc}
\usepackage{mdwlist}
\usepackage{hyperref}

\usetheme{UAmnf}

\setbeamerfont{frametitle}{family=\sffamily\firamedium}
\setbeamercolor{frametitle}{fg=white}
\setbeamertemplate{navigation symbols}{}
\setbeamertemplate{headline}{  
  \leavevmode
   \begin{beamercolorbox}{logo in headline}
	\includegraphics[width=\paperwidth]{UA_mnf_headline}
   \end{beamercolorbox}
}
\setbeamercolor{alerted text}{fg=red!70!black}

\hypersetup{%
  pdftitle={Introduction to Feynman path integrals}
  ,pdfauthor={Gert-Ludwig Ingold}
  ,pdfsubject={Presentation at the Bad Honnef Physics School on
	       Methods of Path Integration in Modern Physics, Bad Honnef, 25.8.2019}
  ,pdfkeywords={Feynman path integrals, propagator, particle in a box,
                particle on a ring, harmonic oscillator}
}

\graphicspath{{./img/}}

\title[Introduction to Feynman path integrals]%
      {Introduction to Feynman path integrals}

\begin{document}

\begin{frame}[t]{}
 \vspace{2.0truecm}
 \begin{center}
   \structure{\LARGE\firamedium Introduction to Feynman path integrals}\\[0.2truecm]

   \vspace{0.1truecm}
   {\Large Gert-Ludwig Ingold}

   {\large Universität Augsburg}

   \vspace{3.4truecm}
   {\scriptsize Bad Honnef Physics School on \textit{Methods of Path Integration in
    Modern Physics}, 25.--31.8.2019}
 \end{center}
\end{frame}

\navtrue

\section{Motivation}

\begin{frame}[t]{Classical mechanics: Hamilton vs. Lagrange}

 \vspace{-0.5truecm}
 \begin{columns}
  \begin{column}[t]{0.5\textwidth}
   \begin{center}
    \textbf{Hamilton formalism}

    Hamiltonian $H(q, p)$

    \vspace{0.2truecm}
    Poisson bracket $\{q, p\} = 1$

    equations of motion $\dot q = \{q, H\}, \dot p = \{p, H\}$
   \end{center}
  \end{column}%
  \begin{column}[t]{0.5\textwidth}
   \begin{center}
    \textbf{Lagrange formalism}

    Lagrangian $L(q, \dot q)$

    \vspace{0.2truecm}
    action $S = \int\text{d}t L(q, \dot q)$

    equation of motion from Hamilton's principle $\delta S = 0$

   \end{center}
  \end{column}%
 \end{columns}

 \begin{columns}
  \begin{column}[t]{0.5\textwidth}
   \begin{center}
    $\Arrowvert$

    canonical quantization $[\hat q, \hat p] = \text{i}\hbar$

    $\Downarrow$
   \end{center}
  \end{column}%
  \begin{column}[t]{0.5\textwidth}
   \begin{center}
    $\Arrowvert$

    ?

    $\Downarrow$
   \end{center}
  \end{column}%
 \end{columns}

 \begin{columns}
  \begin{column}[t]{0.5\textwidth}
   \begin{center}
   Heisenberg, Schrödinger, \ldots (1925)

   Schrödinger equation $\hat H\vert\Psi\rangle =
	  \text{i}\hbar\frac{\partial}{\partial t}\vert\Psi\rangle$ 
   \end{center}
  \end{column}%
  \begin{column}[t]{0.5\textwidth}
   \begin{center}
    Dirac (1933), Feynman (1948)

    \vspace{0.2truecm}
    \fcolorbox{red!70!black}{red!10}{%
       \begin{minipage}{0.75\textwidth}
	\begin{center}
         \alert{Path integral formulation of quantum mechanics}
	\end{center}
       \end{minipage}
    }
   \end{center}
  \end{column}%
 \end{columns}
\end{frame}

\begin{frame}[c]{Why base quantum mechanics\\ on Lagrange formalism?}
 path integral formulation of quantum mechanics \ldots
 \begin{itemize}
  \item \ldots can provide an alternative view on physical problems
  \item \ldots does without operators\\
	for fermionic fields, Grassmann variables are used
\item \ldots is very well suited for relativistic field theories\\[0.2truecm]
	action $S=\int\text{d}^4x\mathcal{L}(\phi, \partial_\mu\phi)$
		 is a Minkowski scalar\\[0.2truecm]
	Hamiltonian density $\mathcal{H}(\phi, \pi)$ corresponds to the
	$0-0$ component of the energy-momentum tensor
 \end{itemize}
\end{frame}

\section{Propagator}

\begin{frame}[t]{}
\end{frame}

\section{Derivation of\\ path integral}

\begin{frame}[t]{}
\end{frame}

\section{Particle in a box}

\begin{frame}[t]{}
\end{frame}

\section{Particle on a ring}

\begin{frame}[t]{}
\end{frame}

\section{Harmonic oscillator}

\begin{frame}[t]{}
\end{frame}
\end{document}
